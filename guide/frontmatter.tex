\frontmatter
\chapter{前言}
本书作为为数不多的涉及Java8+的教材,期望能够帮助读者在更高的平台上(本书放弃了Java5之前的部分特性)
尽快掌握Java语言及其编程技能。本书的目标是在尽量短的时间内,比如48个学时,帮助Java初学者掌握基本的
Java语法和编程思路。

一般的,理工科院校讲授的第一门计算机程序设计语言是C语言,因此大家在学习(自学或者教学)
Java语言时已经具有了一定的C语言基础。而目前常见的Java语言教材或者辅导材料基本上
都是零基础开始讲授Java语言程序设计的,完全无视大家的C语言基础,导致教与学活动中
的很大浪费,也不利于突出Java教学活动的重点。因此,本书不是一本零基础的Java教程,
本书在很多章节将C语言和Java做了对比,引导读者从C语言转换到Java语言上来,作者期望
读者在C语言的基础上能够实现Java语言的快速入门。

\section*{本书的特点}
本书注重培养良好的思维习惯、编程风格,而不是简单的知识传授。很多教材的示例代码过于
随意,可能是作者觉得示例代码过于短小,不值得精心雕琢吧。区别于大多数教材的是,
本书在编写示例代码时,全部经过作者的一手调试,并尽力保证类名、变量名等命名习惯以及
编程风格(包括注释风格)接近工业标准,以帮助读者”先入为主“的了解正确的Java编程姿态。
如果我们学习Java的目的是对接工业化应用,作者认为这样可以事半功倍。

本书注重从实战中学习Java的知识点,本书尽力做到每一个概念、每一个知识点都是可以验证的,
书中的每一个例子都是完整可运行的,都经过作者的
亲自调试。本书的所有示例均可以从\url{https://github.com/subaochen/java-tutorial-examples}
下载,作者也会根据读者的反馈不断完善和调整示例程序。

本书通过思维导图的方式绘出了Java的各个知识点,并针对每个知识点设计了练习题,帮助
读者更好的了解这些知识点及其在实际中的应用方式。

本书的例子均遵循google的编码规范(参考附录),希望读者在阅读本书示例代码的同时能够
直观的感受google编码规范并逐步养成遵守编码规范的好习惯。

\section*{如何阅读本书}

本书不强调从一开始就掌握Java的每个技术细节,而是强调“先跑起来!”,即首先动手写出
一个可以运行的Java应用程序,并理解这个应用程序为什么能够跑起来,然后再逐步扩充到
细节问题。也就是说,在实践中体味和掌握技术细节,在实践中逐步形成良好的编程风格和
思维习惯。

作者建议的阅读方式是:尽快、尽早的下载本书配套的示例代码,在阅读到例题的时候,
直接用Idea打开相应的项目即可查看和运行。同时,建议不要拘泥于作者提供的示例代码,
读者完全可以也应该不断尝试修改并测试。编程能力就是在不断尝试中发展起来的。

本书也特别强调循序渐进的学习路线和知识的前后衔接以及连贯性。虽然Java语言经常存在
一个概念贯穿前后的情况,但是本书强调先掌握这个概念的部分特征,随着学习的深入,将
在后续章节逐步展开,并给出恰当的交叉索引将知识点逐步串联起来。

\iffalse
本教材特别强调学生主动参与的重要性。一般的,首先给出一个可以运行的简单示例,然后
通过不断提出问题的方式,让学生自己“探索”解决问题的方法,最终完成一个相当不错和有些
复杂的程序。通过这种方式将各个知识点串起来,弄清楚程序设计语言为什么这样设计和定义。
看起来一切都是学生自己完成的,这样学生的成就感很强,自学能力和自学的信心都会得到
很大程度的提高。
\fi

本书的章节标题如果是*开头的为较高要求内容,读者可以有选择的阅读,或者作为进阶的
阅读材料。

\section*{你适合阅读本书吗}
本书不是一本零基础的Java语言教程。本书假设读者已经有了一些C语言的基础,掌握了C语言的
基本语法。另外,本书在很多方面比较深入的探讨了Java编程的理念和最佳实践,也可以作为
工程技术人员学习Java语言或者进一步理解Java语言的材料。

\section*{本书的体例}
\subsection*{印刷约定}
\begin{tabular}{|l|l|l|}
    \hline
    字体 & 意义 & 示例 \\
    \hline
    \textsl{AbCd123}斜体 & 文件名、路径名、域名等 & ls -l \textsl{filename}\\
    \hline
    \textbf{AbCd123}加粗 & 在终端输入的命令等 & subaochen\_desktop\% \textbf{su} \\
    \hline
    \texttt{等宽字体} & 示例代码、代码片段等 & \texttt{publc class MyClass...} \\
    \hline
\end{tabular}

\subsection*{图形标识}

本书使用了如下的图形标识帮助读者更好的区分和了解知识点所在:

\bgroup
\def\arraystretch{2.5}%
\begin{tabular}{ll}
\raisebox{-.4\height}{\includegraphics[width=8ex]{imgs/note.png}} & 需要注意的知识点。\\ 
\raisebox{-.4\height}{\includegraphics[width=8ex]{imgs/tip.png}} & 工程实践中常用的技巧。\\ 
\raisebox{-.4\height}{\includegraphics[width=8ex]{imgs/warning.png}} & 容易出错的地方,需要特别小心。\\ 
\end{tabular}
\egroup

\section*{联系作者}
您可以通过我的博客获得最新的消息:\url{http://dz.sdut.edu.cn/blog/subaochen},或者给
我发Email:subaochen@126.com。
本书中的示例代码可以从\url{http://github.com/subaochen/java-tutorial-examples}下载,
也欢迎读者不吝指教,在github提交PR,或者直接发邮件讨论亦可。
您可以在本书的不同章节看到如何获得源代码的相关提示。

\section*{本书是如何写成的}
本书全部使用开源(Open Source)软件完成:
\begin{itemize}
    \item Linux,本书所有稿件和源代码均在Linux下完成。
    \item git,本书的写作过程全程通过git进行版本控制,也借助于git在办公室、家和旅程中实现文档的同步,收益良多。
    \item Lyx(\url{http://www.lyx.org}),优秀的Latex前端可视化工具,最新版本
        (本书写作时是2.2)配合xetex、CTex可以很好的支持中文处理。
    \item graphviz,灵活而强大的代码绘图工具,本书部分流程图是使用graphviz绘制的。
    \item Shutter(\url{http://shutter-project.org}),Linux下面优秀的截图工具。
    \item ArgoUML(\url{http://argouml.tigris.org}),优秀的开源UML工具。
    \item umbrello(\url{https://umbrello.kde.org}),优秀的开源UML工具,本书部分UML图是使用umbrello绘制的。
    \item Dia,优秀的开源绘制工具,堪称Linux下面写作绘图的”瑞士军刀“,本书的大部分
        流程图、框图和UML图都是用Dia绘制的。
    \item Inkscape,优秀的开源矢量图绘制工具,本书的大部分矢量图是用Inkscape绘制的。
    \item vym(\url{http://www.insilmaril.de/vym/}),思维导图绘制工具,本书所有思维导图都是使用vym绘制的。
\end{itemize}

\section*{致谢}
感谢山东理工大学电气与电子学院对本书的资助,感谢各位同仁给予的帮助和指点,
感谢爱妻程玉华承担了大部分家务,让我有充足的时间专心写作;感谢实验室的小
伙伴们,尤其是胡安禄同学和徐千惠同学,帮助画出了部分示例图;感谢李松阳同学
在繁忙的工作之余认真校对,从标点符号到遣词造句都给出了很多具体的意见和建议。

特别的,要感谢我的女儿宿佳敏。写作本书时她正在读高中,作者答应在这个寒假结束
时写完这本书送给她。我做到了,信守一个父亲的诺言。作者坚信,这是最长情的陪伴和
对她学习的最好帮助。

在写作本书过程中,作者查阅了大量Java教材和网络资料,所引用或依据的精彩论述和案例
尽量在文中标出,以便读者参照和比较,同时对原作者表示深深的敬意和感谢!

也感谢所有为开源软件作出贡献的人们!二十年前,当年懵懂的作者决定彻底拥抱开源软件时,
为了安装一个Linux系统曾经不眠不休两天两夜;回头望,开源软件蓬勃发展二十年,
值得欣慰,值得弹冠相庆!没有Linux,没有Latex,没有Lyx,没有Dia,这本书也
不可能如此顺利的完稿。致敬,Open Source!

\hfill 宿宝臣

\hfill 2017年2月

\hfill 于山东理工大学1号实验楼
\mainmatter
